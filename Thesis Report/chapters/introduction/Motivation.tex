\chapter{Introduction}
\label{ch:introduction}
\section{Motivation}
\label{sec:Introduction:Motivation}
The confidentiality of information is one of the most crucial components of data security, and it is vital that only authorized individuals can access sensitive information \cite{al2021ensuring}. Authentication procedures play a crucial role in maintaining information confidentiality by verifying the identity of the user requesting access to secure data \cite{jayarathne2016brainid}. And when it comes to user authentication, the primary goal of user authentication is to deny or confirm an identity claimed by a particular person. Authentication happens in two phases: enrolment and verification \cite{pham2013eeg}. During the first phase, the user has to enroll or register in the system, and therefore user's data is captured and stored in the database. In the second phase, the authenticity of the user's data is checked by matching the user's presented data with his existing data in the database, and based on the similarity of the data, the system grants or rejects the user. 
%The privacy and scrupulousness of such information must be protected by comprehensive and robust access control mechanisms incorporating identification and authentication. %Identification is an important process that allows users to prove that they are who they say they are. It is usually done by providing a unique identifier, such as an email address or user ID. This identifier is then verified by the system and the user is granted access to the system or service they are attempting to use. Identification can also be done through biometric data, such as fingerprints, voice recognition, or facial recognition. The purpose of identification is to ensure that the user is who they say they are and to provide a secure way to access services and systems. Identification can also be used to restrict access to certain resources or services or to ensure that only the right people have access to sensitive information or data. Identification is a critical process in ensuring that user data is secure and that only the right people have access to it.
%Identification is a process that allows an individual to prove what he claims to be by providing unique identifiers such as user ID, username, or email address. On the other hand, authentication is a process primarily used to verify the identity of an individual by verifying the provided credentials such as passwords, tokens, or biometrics. 
%Despite being closely related to each other, the two terms differ in their method of recognizing authorized users.  
At the moment, there are three ways to authenticate a user: through something they know (e.g., password), something they own (e.g., a token or an ID card), and something they are (e.g., fingerprint, eyes, face or other biometric data \cite{grassi2017draft}. 
\smallskip

Knowledge-based authentication is the simplest method, which involves verifying the identity of a user by requesting a password or PIN (personal identity number) known only to the user. Knowledge-based authentication has some advantages, such as passwords being easy to use and maintain. Further, passwords can be revoked easily when compromised. However, passwords suffer from many vulnerabilities, such as complex passwords that are often hard to remember. As a result, users tend to use short and easy-to-remember passwords and reuse them across multiple websites, which exposes the passwords to attackers to breach \cite{lashkari2009shoulder_surfing}. A study by Das \textit {et al.} \cite{das2014tangled} examined several hundred thousand leaked passwords from eleven websites and conducted a password reuse survey based on which it was estimated that 44 to 51\% of users reuse the same password across multiple websites. In their study, the authors also developed a cross-site guessing platform that could guess 
approximately 10$\%$ of the nonidentical password pairs in fewer than ten attempts and approximately 30$\%$ in fewer than 100 attempts. Furthermore, passwords can also be stolen through casual eavesdropping (shoulder surfing) \cite{lashkari2009shoulder_surfing}, or can be guessed using sophisticated hacking algorithms such as dictionary search attacks in which words and word combinations are hashed and then checked for matches against hashed passwords \cite{o2003comparing}. The inherent security vulnerabilities associated with password usage compromise their effectiveness in establishing robust authentication systems.
\smallskip

Possession-based authentication requires the user to possess something such as a token ID to verify their identity. Like passwords, tokens are easy to maintain and can be revoked easily if lost or compromised. A second advantage is that it provides compromise detection since the absence of it is observable (loss of a password, however, does not offer this advantage) \cite{o2003comparing}. While tokens hold some preeminence over passwords in certain respects, they still need to be a foolproof authentication method because they have several security flaws. For instance, tokens are susceptible to theft \cite{authentication_definition} and duplication, meaning that someone might create a counterfeit device \cite{o2003comparing}.  
%Additionally, tokens need to be carried everywhere for authentication which is not feasible \cite{o2003comparing}.
%tokens have to be physically carried everywhere which is not convenient.     
\smallskip

The third form of authentication is biometric-based, relying on distinctive user biometrics for authentication \cite{o2003comparing}. Biometrics can be divided into two classes: Physiological and behavioral. Physiological biometrics are related to the physical features of the human body and therefore differ from person to person. Fingerprints, face recognition, hand geometry, and iris recognition are some examples of physiological biometrics \cite{bhattacharyya2009biometric}. On the other hand, behavioral biometrics refers to an individual's behavior, such as gait, voice, or signature. Unlike passwords and tokens, biometrics do not need to be memorized or physically carried everywhere. They are also unique and cannot be imitated easily. Clearly, biometrics provide more effective authentication mechanisms than passwords and tokens, but biometric-based authentication still needs to be indomitable.  
%However, biometrics also has certain security lapses that attackers have exploited for years.
%But they are prone to forgery as fingerprints can be synthesized artificially, and a copy of the person's face be made if their face has been recorded in some video. 
It is possible to record or photograph biometric information, such as a voice, face, iris, retina, and fingerprint \cite{authentication_definition}. Further, unlike passwords and tokens, biometrics are not easy to replace if they are lost or compromised \cite{o2003comparing}, and the person with specific physical disabilities (e.g., blindness or quadriplegia) cannot use biometric systems, requiring eyes, fingerprints, or gait to authenticate. Each of the authentications, as mentioned earlier methods, has its own merits and weaknesses which need to be addressed. An alternative authentication method is required to overcome existing authentication methods' weaknesses and provide a robust mechanism to verify the identity of users. \smallskip

There has been a rise in interest in using brain activity for next-generation biometric systems to fill in the gaps left by current biometric techniques or
to complement them \cite{arias2021inexpensive}. The technological advances in the last few years have made it possible to obtain brain signals using Electroencephalography (EEG) and utilize the unique characteristics of EEG signals to verify a person's identity \cite{la2014human}. The following are some of the advantages of brainwaves that give them a giant leap over other biometric traits for authentication:

\begin{enumerate}
\item Brain activities cannot be seen from the outside and are therefore impervious to any form of surveillance \cite{arias2021inexpensive}, unlike other biometric traits, for instance, face or gait, which are observable from the outside and can be exploited to identify users without their consent \cite{posteriori_model_adaption}.

\item It is also impractical to steal brainwaves because a person's brain activity is susceptible to their stress and mood, and an aggressor cannot make the victim repeat their mental passphrase \cite{posteriori_model_adaption}. For example, suppose a person is frightened or stressed out. In that case, the brainwaves recorded during the authentication phase will differ significantly from the brain data collected during the person's enrollment into the system. Thus, the system would refuse to grant access if an attacker forces the person to provide his brainwaves.

\item Brainwaves can only be produced by living brain tissue \cite{N400_authentication}. Therefore, brainwaves are a promising candidate for being used as a biometric trait since they can readily handle the main problem of liveness detection in other biometrics \cite{arias2021inexpensive}. 

\item Brainwaves are organically a part of the human body, so even those who are physically disabled can utilize them, unlike with fingerprints or other types of technology, which may not be possible \cite{authentication_definition}. 
\end{enumerate}
\smallskip

\section{Problem Description}
\label{sec:Introduction:Problem Description}
As elaborated in the previous section \ref{sec:Introduction:Motivation}, authentication systems based on brainwaves offer a compelling alternative to conventional authentication systems. However, simply building a brainwave authentication system under the presumption that the chosen algorithm or evaluation metrics confer optimal security is insufficient. Even the most meticulously designed brainwave authentication system may have hidden flaws that are not immediately discernible. As a result, it is crucial to identify and address the specific research gaps in order to make sure that the system being developed provides robust and reliable security. The following research gaps in the field of EEG-based person authentication are the focus of the study described in this thesis, which will be designed to improve the system's high levels of security, performance, and stability.
%However, merely developing any brainwave authentication system is not sufficient enough to prove that our chosen algorithm or the evaluation metrics provides the best security mechanism. Even after developing any brainwave authentication which may seem to be perfect at first glance, it still has flaws and therefore it is imperative to address certain research gaps. 


\begin{enumerate}
\item \textbf{\large Comparative Performance Evaluation and Reporting}
\smallskip

An EEG-based person authentication system's effectiveness relies on feature extraction, data pre-processing, and modeling techniques. Numerous machine learning algorithms such as Linear Discriminant Analysis (LDA) \cite{seha2019new}, Support Vector Machine (SVM) \cite{SVM}, and Na\"{\i}ve Bayes (NB) \cite{naive_bayes} have been proposed and focused on optimizing the Accuracy (ACC) of the system. However, examining the performance of authentication models based on the ACC metric can be flawed if we have an imbalanced dataset \cite{sugrim_robust_metrics}. Other standard metrics to assess an authentication system's performance include False Acceptance Rate (FAR) and False Rejection Rate (FRR). FAR is defined as the proportion of times the system mistakenly accepts the unauthorized person.
On the contrary, FRR gives an overview of instances where the system has denied access to an authorized person. The point where both FAR and FRR are equal is known as Equal Error Rate (ERR) \cite{arias2021inexpensive}. 
\smallskip

Additionally, more than ACC, FAR, and FRR comparisons is required since the specifics of the intrinsic trade-offs that a system must make when implemented are concealed by these standard performance metrics \cite{sugrim_robust_metrics}. These metrics, however, are tied to a specific configuration of the classification threshold. Instead, using Receiver-Operating-Characteristic (ROC) curves to depict results is advisable. These curves plot the relationship between the False Acceptance Rate (FAR) and True Positive Rate (1-FRR), parametrically linked to the threshold value \cite{arias2021inexpensive}. Evaluating existing research on brainwave authentication is complex due to often incomplete metric reporting (frequently only optimized configurations are presented, without ROCs) and variations in samples, algorithms, experimental conditions, and other performance-affecting factors, which are not uniformly reported or accounted for \cite{arias2023performance}.
%\smallskip
%\textbf{ddfw}
%\\

\item \textbf{\large Retraining of Authentication Models} 
\smallskip

A typical brainwave authentication algorithm requires the creation of a unique classifier for each individual. 
Accordingly, these classifiers are trained to recognize the individual designated as 'authenticated' and reject all other users labeled 'rejected.' Although this strategy was initially successful, it faced significant challenges as new users were added to the system. Each new user obligates extensive retraining of the existing classifiers, a step vital for acclimating these classifiers to the unique characteristics of the new user. This computationally demanding repeated retraining raises significant scaling issues. Additionally, it hinders the system's capacity to effectively adapt to real-world scenarios where user bases frequently change, diminishing both its general effectiveness and its usefulness.   
%The addition of each new user obligates a comprehensive retraining of existing classifiers, an action meant to acquaint these classifiers with the newcomer.
%This iterative retraining requirement is computationally intensive and raises substantial scalability concerns. 
%Moreover, it impedes the system's ability to adapt smoothly in dynamic, real-world situations where user bases often fluctuate, thus undermining its overall efficiency and practical applicability.

%The establishment of EEG-based authentication systems employing traditional authentication algorithms necessitates creating a unique classifier for each individual. These classifiers are designed to discern authenticated individuals, designated as 'authenticated', from all other subjects, marked as 'rejected'. This methodology, while initially effective, encounters significant challenges when incorporating new users into the system. The addition of each new user obligates a comprehensive retraining of existing classifiers, an action meant to acquaint these classifiers with the newcomer. This iterative retraining requirement is computationally intensive and raises substantial scalability concerns. Moreover, it impedes the system's ability to adapt smoothly in dynamic, real-world situations where user bases often fluctuate, thus undermining its overall efficiency and practical applicability.

\item \textbf{\large Triviality on Open-Set Scenarios} 
\smallskip

It is essential to consider all the threat case scenarios when developing any authentication system based on brainwaves. The performance of EEG-based authentication systems can be evaluated using two attack scenarios: close-set and open-set scenarios. The close-set scenario assumes that the attacker is enrolled in the system and, therefore, part of the system, while the open-set considers the attackers who are unknown to the system. The open-set scenario provides a more realistic approach since the attacker is not guaranteed to be always known to the system. Moreover, in the context of EEG-based authentication, the presumption that the authentication systems have already encountered the attacker is unrealistic since the authentication systems typically do not have access to the brain signals associated with the attacker\cite{arias2023performance}. Hence, the authentication systems must be able to identify and reject the known attackers as well as the attackers, completely unknown to the system. %But unfortunately, more emphasis has been put on the close-set scenarios while neglecting the security implications of the latter threat scenario. 
Regrettably, most studies on EEG-based authentication have focused primarily on close-set scenarios, often overlooking the security ramifications of the open-set scenarios.

\item \textbf{\large Lack of Research on Inter-Session variability} 
\smallskip

Most of the research on brainwave authentication is conducted by utilizing brain signals, usually collected during a single EEG recording session. Researchers would often split the single-session EEG data for training and testing the effectiveness of the authentication system. However, brain signals can be impacted due to the person’s surrounding environment or the individual’s state of mind. As a result, an extensive study must be conducted on multi-session EEG data where the robustness of the authentication system should be tested on sessions conducted on different days to investigate if the inter-session variability among users can be accounted for a significant drop in the system’s performance. Unfortunately, this crucial issue has been not been addressed by most of the researchers working on brainwave authentication systems.

\item \textbf{\large Reproducibility of Implementation} 
\smallskip

It is also seen in EEG studies that the parameters of the pre-processing procedures, the toolboxes utilized, and implementation techniques are often hidden or reported in a very abstract manner \cite{moabb}. This lack of transparency often propels researchers to spend considerable time trying to reproduce the results reported by state-of-the-art (SOA) proposals. As a result, the process of replication and advancement in brainwave authentication is impeded due to the opaque style of reporting followed within the scientific community. 

\item \textbf{\large Benchmarking Datasets} 
\smallskip
 
Although many studies are available on brainwave authentication, there is still a glaring shortage of open EEG datasets in the scientific community since most researchers chose to keep the EEG data private. Furthermore, the majority of EEG datasets that have been made available to the public involve a small number of participants (N<=30), including studies such as \cite{simoes2020bciaut, hubner2017learning, guger2009many_paper, toffolo2022evoking}. These small-size datasets do not provide a complete picture of the real-world performance of brainwave authentication systems, and the results generated by utilizing those datasets could be highly optimistic as they do not capture the entire spectrum of EEG variability across a larger population. Additionally, the researchers developing those datasets rarely share their source code and data, hindering the reproduction of the results needed to test new algorithms \cite{moabb}.  
\end{enumerate}
In conclusion, because of the extreme differences in the experimental approach employed by various researchers, it is difficult to assess the actual research progress on brainwave authentication. In order to tackle this issue, I address the research question:
\\
\\
\textit{How do state-of-the-art (SOA) EEG-based authentication models compare when evaluated under the same conditions?}   
\smallskip

\section{Solution Overview}
\label{sec: Introduction:Solution Overview}
In the preceding section \ref{sec:Introduction:Problem Description}, we discussed the intricate challenges this study aims to resolve. In this section, we provide a preliminary outline of the proposal that is being contemplated. This thesis aims to develop a robust and scalable benchmarking framework where the performance of different SOA authentication algorithms can be compared, utilizing pertinent evaluation metrics. The following points offer a concise overview of our suggested solutions tailored to address the research questions articulated in the previous section. 
\begin{enumerate}
\item Our study strategically employs evaluation metrics such as TPR, FAR, FRR, EER, and ROC-Curves, all of which ensure unbiased results—particularly ROC-Curves, less sensitive to imbalanced datasets \cite{sugrim2019robust}. We also report FRR corresponding to FAR at 1$\%$, which is essential to balance the system's security and usability. Lower FAR correlates to security's enhanced security measures while FRR pertains to the system's ease of use \cite{arias2023performance}. Therefore, it is imperative to ascertain whether lowering the FAR threshold to increase security may unintentionally render the system less usable. The utilization of all the above mentioned evaluation metrics in our study, provides an effective solution to the first research question outlined in the previous section.     

%Our research strategically employs evaluation metrics such as Equal Error Rate (EER), Receiver Operating Characteristic (ROC) curve, False Acceptance Rate (FAR), False Rejection Rate (FRR), and True Positive Rate (TPR) that ensure unbiased outcomes. We additionally compute the FRR at a 1$\%$ FAR, which is essential to gauge both the security and usability of our authentication system. The FAR encapsulates the system's security measure, while the FRR represents the system's usability. Consequently, there is a necessity to strike a balance between these two metrics. It is crucial to determine whether a reduction in the FAR threshold to bolster security would inadvertently render the system less usable. The utilization of these evaluation metrics provides an effective solution to the first research question delineated in the previous section. 

\item Our approach employs Siamese Neural Networks (SNN) alongside state-of-the-art (SOA) algorithms. SNN is a specific type of neural network with two or more identical sub-networks working in tandem. These concurrent sub-networks are trained with the same hyperparameters to generate the embedding in latent space. Such embedding serves as compact, representative vectors of the input data. This parallel configuration is then utilized to ascertain the similarity among the inputs by comparing their feature vectors \cite{fallahi2023brainnet}. One of the most significant advantages of using SNN is that it mitigates the problem of retaining once a new user is enrolled into the system. Rather than retraining the entire model each time a new user is registered, SNN can generate a unique embedding for the newcomer and compare it to the existing ones, thus reducing computational time significantly.

\item As discussed in the previous section, understanding both close-set and open-set scenarios is essential to determining the resilience and applicability of brainwave authentication systems. Therefore, our research extends beyond the close-set scenarios and greatly emphasizes open-set scenarios. Open-set scenarios present a unique challenge as the system identifies and rejects the attacker's brain signals that it has not previously encountered. As a result, we evaluate each authentication approach under both threat case scenarios, including the SNN.

\item The foundation of our research relies on four publicly accessible EEG datasets. It was relatively straightforward to acquire single-session datasets; however, finding appropriate multi-session datasets proved much more difficult. Despite encountering some multi-session datasets, only a few satisfy the stringent policies set for our study to be used for benchmarking. Section \ref{sec:Solution Approach:Survey Open Datasets} covers the factors that guided us to choose our datasets for the analysis in great detail. After analyzing a handful of multi-session datasets, we narrowed down our selection to one particular dataset, which offered three EEG recording sessions, each session conducted at an interval of seven days. As a result, our study utilizes three single-session datasets and one multi-session dataset for evaluating the performance of various authentication algorithms.

\item Our proposed approach has been specifically designed to meet the needs of researchers active in brainwave authentication. One of the main objectives of our study is to build a framework that should alleviate the time-consuming processes of pre-processing, feature extraction, parameter selection, and classification. The framework's adaptability allows it to integrate with the new data provided by the researchers seamlessly. Our framework significantly reduces the time burden for researchers and offers essential guidance in determining the optimal parameters for their studies.

%Not only does our framework save considerable time for researchers, but it also provides them with valuable guidance in identifying the optimal parameters for their studies. This feature, we believe, will prove instrumental in advancing brainwave authentication research.    
%Our proposed framework is tailored to meet the needs of researchers active in the field of brainwave authentication. The intent behind its creation is to alleviate the time-consuming and intricate processes of pre-processing, optimal parameter selection, and authentication model building. The flexibility of the framework allows it to seamlessly integrate with the data provided by the researchers. Upon receipt of the EEG data, our tool undertakes all the pre-processing, feature extraction, and classification tasks. Subsequently, the tool offers a comprehensive performance assessment of the dataset across multiple algorithms, using evaluation metrics such as Equal Error Rate (EER), Area Under the Receiver Operating Characteristic Curve (AUC), and False Rejection Rate (FRR) at 1$\%$ False Acceptance Rate (FAR).
%Not only does our framework save considerable time for researchers, but it also provides them with valuable guidance in identifying the optimal parameters for their studies. This feature, we believe, will prove instrumental in advancing brainwave authentication research.

%Our framework is made for researchers working in brainwave authentication. This will save their precious time, spent on pre-processing, finding optimal parameters and building authentications models. Our proposed framework is flexible enough that it will accomodate the data of reserachers in such a way that the reserachers will provide their EEG data and our tool will perform all the pre-rpocessing, feature extraction and classfication on their given. The tool finally provides a performance overview of their dataset on different algotithms with evaluation metrics like EER, Area Under Curve (AUC) and FRR at 1$\%$ FAR. This will not only save the lot of time of the reseachers but can also guide them in selecting the best parameters for their reserach. 

\item Each dataset selected for our research incorporates data from a sizeable population (N>=25). The collective number of participants from our selected datasets comes out to be 195, which is approximately a quadruple increase over earlier studies in brainwave authentication. Designing an authentication framework based on such a large population allows for better coverage of EEG variability throughout a broader spectrum. Consequently, the results derived from our study, which includes 195 participants, will be more reliable and generalizable as they are less likely to show bias or overly optimistic expectations.     
\end{enumerate}
\smallskip

\section{Thesis Structure}
\label{sec: Introduction:Thesis Structure}
In this chapter, we have articulated the primary motivations driving this study, outlined the research questions we want to answer, and given a rough outline of the approach we want to take. The next chapter will deal with brainwave authentication's foundations, including the background and core concepts of brainwave authentication, such as common EEG EEG devices, data acquisition procedures, data processing methods, and authentication algorithms. Chapter \ref{ch:Related Work} will be focused on the current proposals and state-of-the-art research works compared to the challenges considered in this thesis. In the subsequent chapter \ref{ch:Solution Approach}, we offer an in-depth analysis of the surveyed open datasets. Additionally, we present a short overview of the workflow employed in our benchmarking framework in this chapter. The practical implementation and mythologies devised by us to build the framework will be described in detail in chapter \ref{ch:Framework}. The evaluation aspects of this study will be discussed in Chapter \ref{ch:Evaluation}. This chapter assesses the performance of various authentication algorithms, the outcomes of the research, and a comprehensive analysis of the evaluation results. In the forthcoming chapter \ref{ch:Discussion}, an analysis will be conducted on the replicated results of our framework through a comparative examination of our research with prior investigations in brainwave authentication. Chapter \ref{ch:Conclusion and Future Works} concludes with a discussion of this study's findings and potential future enhancements to the proposed application.

%In the subsequent chapter \ref{ch:Discussion}, we will discuss the reproduced results of our framework by comparing our work with the existing studies in brainwave authentication. Finally, 


%Further in chapter \ref{ch:Solution Approach}, we provide an insight on the surveyed open datasets and also the illustrate the workflow of our benchmarking framework. 




\label{sec:introduction:start}

% In this section, we briefly discuss how to set up your thesis to work with the
% \acs{template} template.
% This is a four step program:
% \begin{enumerate}
% 	\item choosing a language,
% 	\item setting up the title page of your thesis,
% 	\item writing its abstract, and 
% 	\item writing the thesis.
% \end{enumerate}dfdfkjd

% The \acs{template} template can set up your thesis to use either English or
% German language options.
% By default, English is used.
% If you want to use German, edit the line of \emph{thesis.tex} that reads
% \verb+\documentclass[]{upb\_cs\_thesis}+ and write \emph{german} (lower case!)
% into the pair of brackets.

% What you need to set up the title page are a \emph{title}, your \emph{own
% name}, the \emph{type} of the thesis and the academic \emph{degree} you aim
% for, the name of your \emph{supervisor} and his or her \emph{research group's
% name}, as well as a submission date.
% Setting up your thesis's title page, by editing the main file
% \emph{thesis.tex}. 
% In the section marked ``your thesis title, [\dots],'' you find certain
% commands to which you pass the information listed above.
% Pass your thesis's title to the \verb+\thesis+ command, and your own name to
% the \verb+\author+ command.
% The type of your thesis is passed to the \verb+\thesistype+ command.
% You can comment in or out any of the example types given in the file, or pass
% a string of your own choice.
% Similarly, for the academic degree choose any of the given examples or pass
% your own choice.
% Your supervisor's research group's name is passed to the \verb+\researchgroup+
% command, while his or her name goes into the \verb+\supervisor+ command.
% The submission date of your thesis is passed to the \verb+\submissiondate+
% command; you can pass the current date via the \verb+\today{}+ command, \ie{}
% not changing the template file.

% If you already know your thesis's abstract, write it down in file
% \mbox{abstract.tex}.
% You can leave the file unchanged, in which case this template documentation's
% abstract is used instead until you change file \mbox{abstract.tex}.

% Finally, start writing your thesis.
% Your texts should go into the chapters directory, as described in
% Section~\ref{sec:introduction:folders:chapters}.
%The most crucial factor to consider when ensuring information confidentiality is %authentication. 

%cghchgchg

%\section{Organisation: Directories}
\label{sec:introduction:folders}

The \acs{template} template assumes a simple directory structure.
It has a main directory that holds all files relevant to the template itself.
Three complementary directories hold your texts, figures and appendices.



\subsection{Main Directory}
\label{sec:introduction:folders:main}
Your main directory contains the four sub-directories
\begin{enumerate}
	\item \texttt{chapters},
	\item \texttt{figures},
	\item \texttt{pretext}, and
	\item \texttt{appendices}.
\end{enumerate}
These directories are further described in separate sections. The main
directory also contains a couple of files: 
\begin{description}
	\item[\texttt{abstract.tex}] This file holds the abstract of your
	thesis.

	\item[\texttt{appendix.tex}] This file groups your appendices together.
	The file is \verb+\input+ into the main file \mbox{thesis.tex}.
	The appendices themselves should go into the \texttt{appendices}
	folder, and be included into \mbox{appendix.tex} via the \verb+\input+
	command.

	\item[\texttt{body.tex}] This file contains the main text of your
	thesis.
	The file is \verb+\input+ into the main file \mbox{thesis.tex}.
	The texts themselves should go into the \texttt{chapters} folder, and be
	included into \mbox{body.tex} via the \verb+\input+ command.

	\item[\texttt{commands.tex}] Your custom \LaTeX{} commands should go
	into this file.
	The file is \verb+\input+ into the preamble of the main file
	\mbox{thesis.tex}.
	By default, \mbox{commands.tex} defines commands for common
	abbreviations, and some environments.
	See Section~\ref{sec:packages:commands} for further information.

	\item[\texttt{literature.bib}] Your bibliographic references should go
	into this file.
	See Section~\ref{sec:guide:bibliographies} for further information.

	\item[\texttt{Makefile}] This file can make your life easier. It
	provides means for simple and complex compilation processes. For
	details, see Section~\ref{sec:makefiles:main}.

	\item[\texttt{packages.tex}] This is where you include \LaTeX{}
	packages.
	The file is \verb+\input+ as the first file into the preamble of the
	main file \mbox{thesis.tex}.
	By default, it loads the auxiliary packages as described in
	Section~\ref{sec:packages:auxiliary_packages}.

	\item[\texttt{thesis.tex}] This is the main file for your thesis.
	It defines the document class, includes the preamble, and loads files
	\mbox{body.tex} and \mbox{appendix.tex}.
	It also sets up the title page, legalities, and the table of contents.

	\item[\texttt{upb\_cs\_thesis.cls}] This is defines the document class
	used by the \ac{template} template.
	See Chapter~\ref{ch:class_file} for details.
\end{description}



\subsection{Chapters Directory}
\label{sec:introduction:folders:chapters}
This directory is where the \mbox{.tex} files of your thesis go.
It is advisable to have one file per chapter.
For long chapters, you should create additional files for smaller bits of text,
\eg{} on the section or even subsection level.
The low level files should then be \verb+\input+ into the respective chapter's
\mbox{.tex} file, while the chapter's \mbox{.tex} file should be \verb+input+
in main directory's \mbox{body.tex}.



\subsection{Figures Directory}
\label{sec:introduction:folders:figures}
This directory is where your graphics go.
See Section~\ref{sec:guide:graphics} on how to include graphics in your thesis.
By default, this directory contains the university's logo
(\texttt{uni-logo.pdf}), a \texttt{Makefile}, and the \texttt{template\_files}
sub-directory. 

The university's logo is used on the title page.
The \emph{Makefile} of the \emph{figures} directory converts some types of raw
data into graphics; see Section~\ref{sec:makefiles:figures} for details.
Finally, the \emph{template\_files} sub-directory contains the
\texttt{tikz\_template.tex} file, that can be used to create
tikz\footnote{\TeX{} Ist Kein Zeichenprogramm, a \LaTeX{} package for
programming graphics} graphics that the figures directory's \emph{Makefile} can
properly process.



\subsection{Pretext Directory}
\label{sec:introduction:folders:pretext}
This directory contains two files \texttt{erklaerung.tex} and
\texttt{titlepage.tex}.
The former contains a legal statement, the latter defines the layout of your
titlepage.
Typically, you do not need to edit these files.
You fill in the details of your title page as described in
Section~\ref{sec:introduction:start}.



\subsection{Appendices Directory}
\label{sec:introduction:folders:appendices}
This directory is where your thesis's appendices go.
The files should be \verb+\input+ into your thesis via the \mbox{appendix.tex}
file.
By default, this directory is empty.

